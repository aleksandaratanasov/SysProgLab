\chapter{Examples}
%\addcontentsline{toc}{chapter}{Examples}
\paragraph{}
  This chapter gives a couple of examples that the user can test to see how the compiler performs on his/her platform.
  
	\section{Bubble sort}
	\paragraph{}
  		Sorting algorithms are an essential part of programming and bubble sort is the most simple (and also most inefficient) way to do it:
    		\begin{lstlisting}[frame=single]
 int[5] arr;
 int i;
 int j;
 int temp;

 (* array that needs to be sorted *)
 arr[0] = 10;
 arr[1] = -2;
 arr[2] = 3;
 arr[3] = 7;
 arr[4] = 0;

 (* display the unsorted array *)
 i = 0;
 while(i<5){
   print(arr[i]);
   i = i + 1;
 };

 i = 0;
 while(i<5){
   j = 0;
   while(j<4){    
     if(arr[j]>arr[j+1]){
       temp = arr[j+1];
       arr[j+1] = arr[j];
       arr[j] = temp;
     }
     else
       temp = 0;
     j = j + 1;
   };
   i = i + 1;
 };

(* display the sorted array *)
 i = 0;
 while(i<5){
   print(arr[i]);
   i = i + 1;
 };
		\end{lstlisting}
				
	\paragraph{}
	  	The code above can easily be expanded by allowing user interaction. By using the \textit{read()} statement in a loop the user will be able to enter whatever numeric values he/she wants to sort. The language's grammar however does not allow generation of an array of a user-defined size mostly because all declerations have to be placed at the beginning of the code. It was mostly used to test array access and nesting of conditional statements.
	  
	\section{Fibonacci numbers}
	\paragraph{}
	  A good way to test not only loops but the out of range error for numeric values that are too large to be stored inside a long int are the Fibonacci numbers. In the code below the first 45 of the series are displayed:
	  \begin{lstlisting}[frame=single]
 int i;
 int j;
 int c;
 int n;
 int count;
 int d;
 i = 0;
 j = 1;
 n = 45; (* first 45 numbers in the series *)
 count = 0;
 c = i+j;
 print(i);
 print(j);

 while(count < n-2) {
   d = j+c;
   print(d);
   j = c;
   c = d;
   count = count + 1;
 };
	  \end{lstlisting}
	  
	\section{Emulation of for-loop with post increment}
	\paragraph{}
		Each loop can be converted to the other types of loops. The while-loop that is provided by the language here allows a simple emulation of a for-loop with post increment:
		\begin{lstlisting}[frame=single]
 int i;
 int[10] arr;

 (* Emulation of the for(i=0; i<10; i++) statement *)
 i = 0;
 while (i < 10){
   print(i);
   arr[i] = i*10;
   i = i + 1;
 };
		\end{lstlisting}
	  
	\section{Partial test of logical operators}
	\paragraph{}
		Due to the presence of conditional statements checking if the logical operators are working properly is very important. The code below offers a small portion of the test that was done while testing the compiler:
	 	
	 	\begin{lstlisting}[frame=single]
 int a;
 a = 10;
 
 print(a);

 if(a > 10)
   print(1)
 else
   print(0);

 if(a <=> 10)
   print(1)
 else
   print(0);

 if(a < 10)
   print(1)
 else
   print(0);

 if(!(a = 10))
   print(1)
 else
   print(0);

 if(!(a<=>10))
   print(1)
 else
   print(0);

 if((a < 100) & (a = 10))
   print(1)
 else
   print(0);

 if((a = 100) & (a = 10))
   print(1)
 else
   print(0);

 if((a = 10) & !(a < 5) & (a <=> 10))
   print(1)
 else
   print(0);
	 	\end{lstlisting}