%\rhead{hskac}
%\lhead{Parser}
\chapter{Parser}
%\addcontentsline{toc}{chapter}{Parser}
\label{chapter:parser}
\paragraph{}
	The \textit{parser} is responsible for verifying if the \textit{tokens} generated by the scanner obey the grammatic rules of the language and also if the types are correct. After this two-stage verification is done the final code is generated and written to a file that can be run by the provided interpreter.
	
	\section{Generation of the abstract syntax tree}
	\paragraph{}
		The \textit{abstract syntax tree} (AST) represent the program's structure based on the source file written by the user. Each node in the \textit{AST} is derived from the \textit{Node} class, which represents all terminal symbols in the language's grammar: PROG, DECLS, DECL, ARRAY, STATEMENTS, STATEMENT, EXP, EXP2, INDEX, OP\_EXP and OP. PROG is the root of the tree and all the nonterminals are attached as leaves to EXP2 and OP such as identifiers, brackets and all the operators (see \textit{Figure \ref{fig:ast}}).
		
	\paragraph{}
		After a major bug was discovered during the code generation stage the design had to be changed leading to the implementation of a template list used by DECLS and STATEMENTS containing all subsequent DECL and respectively STATEMENT objects. The way it works is as follows: first an initial DECLS or STATEMENTS node is created and attached to the root of the \textit{AST}. Each of the following declerations/statements is attached to a list in either DECLS or STATEMENTS depending on the situation. By doing so an exact representation of the source code is generated with all the nestings in it and is also well suited for the top-down parsing applied in this compiler.
		
	\begin{figure}
    		\centering
    		\label{fig:ast}
    		\includegraphics[scale=.5]{parser_ast}
    		\caption{\textit{AST} used for parsing the tokens}
    \end{figure}
    
    \section{Checking types}
    \paragraph{}
    		Once the \textit{AST} is completed the type-check stage begins during which each node is assigned a type based on the types of its children (see \textit{Table \ref{tab:types}}). It was decided that adding additional types for the various statements and expressions improved the readability of the code and also allowed a simple switch-case structure to decide what exactly is to be done based on those types.
    		
    		\begin{table}
	    		\begin{tabular}{|l|c|}
	    		\hline
	    		\multicolumn{2}{|c|}{\cellcolor[gray]{0.9}Common types} \\ 
	    		\hline
	    		NO\_TYPE & assigned to abstract nodes in the \textit{AST}: PROG, DECLS, STATEMENTS etc. \\ 
	    		\hline 
	    		ARRAY\_TYPE & assigned if an array is detected \\ 
	    		\hline 
	    		INT\_TYPE & assigned if an integer is detected \\ 
	    		\hline \hline
	    		\multicolumn{2}{|c|}{\cellcolor[gray]{0.9}Operator types} \\ 
	    		\hline
	    		OP\_PLUS\_TYPE & + \\ 
	    		\hline 
	    		OP\_MINUS\_TYPE & - \\ 
	    		\hline 
	    		... & ... \\ 
	    		\hline 
	    		OP\_NOT\_TYPE & ! \\ 
	    		\hline 
	    		OP\_AND\_TYPE & \& \\ 
	    		\hline \hline
	    		\multicolumn{2}{|c|}{\cellcolor[gray]{0.9}Statement types} \\ 
	    		\hline
	    		STATEMENT\_ASSIGN & assignment (e.g. x=1;, x[0]=1; etc.) \\ 
	    		\hline 
	    		STATEMENT\_PRINT & print (e.g. print(x);, print(123); etc.) \\ 
	    		\hline 
	    		STATEMENT\_READ & read (e.g. read(x);) \\ 
	    		\hline 
	    		STATEMENT\_IF\_ELSE & if-else (e.g. if(...) ... else ...;, if(...){...}else ... etc.) \\ 
	    		\hline 
	    		STATEMENT\_WHILE & while loop (e.g. while(...) ...; or while(...){...};) \\ 
	    		\hline 
	    		STATEMENT\_BLOCK & multiple statements block ({...}); used only with if-else or while statements \\ 
	    		\hline \hline
	    		\multicolumn{2}{|c|}{\cellcolor[gray]{0.9}Expression types} \\ 
	    		\hline
	    		EXPRESSION\_BRACKETS & stacked expression (e.g. (x+2)) \\
	    		EXPRESSION\_IDENTIFIER & calling an identifier (e.g. x or x[0]) \\ 
	    		\hline 
	    		EXPRESSION\_INT & numeric (e.g. 0,1,2 etc.) \\ 
	    		\hline 
	    		EXPRESSION\_NEGATIVE & negating a value (e.g. -x, -2 etc.) \\ 
	    		\hline 
	    		EXPRESSION\_NOT & negating a logical expression (e.g. !(x \textless 1) etc.) \\ 
	    		\hline
	    		\end{tabular}
	    	\caption{Types assigned and checked during the type-check stage of the parsing process}
    		\label{tab:types}
    		\end{table}
    		
    	\section{Code generation}
    	\paragraph{}
    		Before the improvement in the parser's design using lists was introduced a major bug was discovered. Initially it was decided that once a statement was generated and its type was verified the code generation for that statement could also be launched after which the statement was to be deleted in order to preseve memory for parsing large programs. However due to the fact that a statement can contain to a block of statements this lead to an unwanted shifting of the conditional statement when generating the code. This shifting created infinite loops and incorrectly evaluated if-else statements. The introduction of the desing based on lists resolved the issue elegantly, inproved the code's readability but also added a memory overhead due to the fact that the complete \textit{AST} is stored in the main memory until the parsing is completed.
    	\paragraph{}
    		The design based on lists for storing DECL and STATEMENT object allows recursive access to each element and calling its \textit{typeCheck()} and \textit{makeCode()} methods.
    		
    \section{Error handling}
    \paragraph{}
    		Unlike the scanner the parser has to have an extensive error handling system for both semantic and type errors. The parser has to stop upon any error unlike the token, which continues generating tokens unless they contain an unknown character or a numeric value is out of range. The user has to be given not only the exact location where the error has occured (line and row) but also after which token and what is the cause of the interruption of the parsing process. Below two examples are given to illustrated:
    		
    		\begin{description}
    			\item[Example for type error] - the last line of the code causes the parser to stop since we try to access the variable \textbf{b} not as an array but as a normal variable, which is a type conflict:
    				\begin{lstlisting}[frame=single]
 int a;
 int[2] b;
 a = 10;
 b[0] = 1;
 b = 9;
				\end{lstlisting}
			The resulting error message is \textit{Type error at [5:3] ("b") :Incompatible types}.
			\item[Example for semantic error] - a common mistake is forgetting to close a bracket as shown in the last line as part of the \textit{print()} statement:
			\begin{lstlisting}[frame=single]
 int a;
 a = 10;
 
 print(a;
			\end{lstlisting}
			This time we get both a semantic error in the form of \textit{Parse error at [4:2] ("print") :Missing ')'}, which points at the missing bracket and also a type error \textit{Type error at [4:2] ("print") :Identifier not defined} due to the fact that the missing bracket also leads to an incomplete type-check.

    		\end{description}
    		    		
    	\section{Buffering of output file}
    	\label{sec:bufferout}
  	\paragraph{}
  		Before code for a particular part of the parsed source code is written to a file it is written to a buffer. The writing buffer is similar to the reading one in that it uses multiple buffers (see \textit{Section \ref{sec:multibuff}}). In this case following three buffers are part of the ring buffer:
  		\begin{description}
  			\item[Current content] - stores the current interpreter code as a result of the code generation
  			\item[Write to disk] - stores the interpreter code that will be written to a file on the filesystem; a buffer storing current content can become a buffer used to write to the output file only when it is full
  			\item[Hot swap] - a single buffer in a waiting state; once the buffer which stores the current content is full it is replaced with this one
  		\end{description}
  	\paragraph{}
		The destructor of the buffer responsible for reading the file closes the file once the reading is completed or an error has occurred.
    		
	\section{Writing output file}
	\paragraph{}
		The interpreter file that will be run by the interpreter needs to be created and written to via the \textit{open()} function using the following flags:
  		\begin{description}
  			\item[O\_DIRECT] - see same flag in \textit{Chapter \ref{chapter:scanner}}
  			\item[O\_WRONLY] - since data will only be written to the interpreter file we restrict the access to it by using write-only mode
  			\item[O\_CREAT] - create the output file
  			\item[O\_TRUNC] - in our case the file might already be present and by using this flag it will be reset to a length of 0 excluding the possibility of concatenating new content to the old one
  			\item[S\_IRUSR] - gives the user reading permissions to the output file
  			\item[S\_IWUSR] - gives the user writing permissions to the output file.
  		\end{description}