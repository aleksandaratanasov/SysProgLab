%\rhead{hskac}
%\lhead{Introduction}
\chapter{Introduction} 
%\addcontentsline{toc}{chapter}{Introduction}
\paragraph{}
	This document describes the internal workings of the hskac (short for \textbf{Hochschule Karlsruhe Compiler}), which is assigned as a lab exercise part of the lecture on \textit{operating systems}. An overview of the documentation can be seen below:
\begin{description}
\item[Scanner] - takes a file containing a user's source code and tokenizes it
\item[Parser] - using the extracted information from the scanner an \textit{abstract syntax tree} is generated that checks if the user's source code represents a meaningful program\footnote{In this documentation a program is considered \textit{meaningful} if it obays a set of given rules (language grammar and types)} after which a code file is created that can be executed in the provided by the teacher interpreter.
\item[Multithreading] - a modification of the scanner for faster processing of the input file and tokenization
\item[Tools] - a short description of the tools that were used in the process of creating the compiler
\item[Examples] - provides several examples that can be used as a test or simply to see how the compilation and interpretation processes work.
\end{description}
Each chapter will also point out some of the problems that were encountered during the development.
\paragraph{}
	There are three assignments along with the documentation on each of those that need to be accomplished in order to pass the lab.
\begin{enumerate}
\item Create a scanner (incl. documentation)
\item Create a parser (incl. documentation)
\item Add multithreading support to the scanner (incl. documentation)
\end{enumerate}

\paragraph{}
	The compiler that is described here implements the language grammer given to the students during the summer term of 2011. Additional documentation is available generated using \textit{doxygen}. It was not included here due to its size (more than 100 pages).