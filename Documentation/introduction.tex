%\rhead{hskac}
%\lhead{Introduction}
\chapter{Introduction}
\paragraph{}
	This document describes the internal workings of the hskac (short for \textbf{Hochschule Karlsruhe Compiler}), which is assigned as a lab exercise part of the lecture on \textit{operating systems}. It has the following structure:
\begin{description}
\item[Scanner] - takes a file containing a user's source code and tokenizes it (more on the process will be revealed in \ref{chapter:scanner}).
\item[Parser] - using the extracted information from the scanner an abstract syntax tree (AST) is generated that checks if the user's source code represents a meaningful program\footnote{A program is considered \textit{meaningful} if it obays a set of given rules (language grammar and types)} after which a code file is created that can be executed in the provided by the teacher interpreter.
\item[Multithreading] - a modification of the scanner for faster processing of the input file and tokenization
\item[Tools] - a short description of the tools that were used in the process of creating the compiler
\end{description}
Each chapter will also point out some of the problems that were encountered during the development.
\paragraph{}
	There are three assignments along with the documentation on each of those that need to be accomplished in order to pass the exam.
\begin{enumerate}
\item Create a scanner (incl. documentation)
\item Create a parser (incl. documentation)
\item Add multithreading support to the scanner (incl. documentation)
\end{enumerate}